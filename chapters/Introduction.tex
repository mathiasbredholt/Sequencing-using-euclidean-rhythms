\chapter{Introduction}

Music software in the form of a \textit{Digital Audio Workstation} (DAW) is widely available within the field of music production. A DAW is a graphical mixing program that enables the composer to arrange audio recordings, apply audio effects and use software synthesizers to create music. The effect- and synthesis parameters can be automated, which allows for complex and precise musical operations that would be impossible to realize by any other means\cite{roads}[p. 374]. But as Miller Puckette, the creator of the music software \textit{PureData}, says in his lecture \textit{Design choices for computer instruments and computer compositional tools} (2012):
\begin{displayquote}
90\% of the time, when making computing music, you're really doing office work. Computers are things which are designed for financial transactions and ledgers and also they are designed for making missile trajectories. (..) They are not things that were ever designed to make music. \cite{puckette}
\end{displayquote}
The interesting point is that computers, though not designed for musical purposes, are still the most versatile and widely used tool for electronic music production. While dedicated music hardware is still being continuously developed and improved, it seems impossible to create hardware that exceeds, or even matches, the capabilities of music software running on a computer. Nevertheless, it seems that many companies within the music hardware industry are trying to do just that. Reading Puckette's quote, one could think that the reason behind the industry's approach, is that many musicians find the "office work" of music software tedious and non-intuitive. In this project another approach is taken. Instead of building hardware that can replace the computer, the purpose of this project is to combine the best of both worlds, by designing music hardware that supplements and cooperates with the computer.\\
Specifically, the hardware designed in this project is a \textit{sequencer} - a programmable device that is able to reproduce a sequence of musical events. The sequencer uses euclidean rhythms, an algorithm for generating rhythms discovered by Godfried Toussaint in his 2005 paper \textit{The Euclidean Algorithm Generates Traditional Musical Rhythms}. This form of sequencing derives from \textit{algorithmic composition} - a composition technique developed through the 20th century, where the music material is generated by an algorithm. As most popular DAWs has very limited capability with this technique, there is a potential for opening up new interesting possibilities for composers working with electronic music. The project is carried out in collaboration with the company \textit{Kenton Productions}, which has provided the visual design of the hardware and app.

% \begin{displayquote}
% Since I have always preferred making plans to executing them, I have gravitated towards situations and systems that, once set into operation, could create music with little or no intervention on my part. That is to say, I tend towards the roles of planner and programmer, and then become an audience to the results"
% \end{displayquote} - Brian Eno \cite{ENO}

% the gestures needed for controlling music software does not always harmonize with our musical cognition, which limits the composition and live performance of electronic music, to a very unresponsive interaction with technology \tdr{kilde}.
% Musicians often come around this issue by using a lot of expensive music hardware \tdr{kilde}, but along with the rapid evolution of music software, there is a potential for accelerating music creation and performance by improving the tools for interacting with music software on a computer.

% laptop, and thereby exceeding the capabilities of music production on dedicated music hardware \tdr{lidt søndagskringlet}.

% As the composer Igor Stravinsky wrote in his \textit{Autobiography}:

% \begin{displayquote}
% It is a thousand times better to compose in direct contact with physical medium of sound than to work in abstract medium produced by one's own imagination
% \end{displayquote}

% sequencing
% algorithmic composition


% generation of rhythms and problems of notation s. 356



% As computers are designed for non-musical purposes \tdr{MPlecture},


% the gestures needed for controlling music software does not always harmonize with our musical cognition, which limits the composition and live performance of electronic music, to a very unresponsive interaction with technology \tdr{kilde}.
% Electronic musicians often come around this issue by using a lot of expensive music hardware \tdr{kilde}, but along with the rapid evolution of the laptop computer,  there is a potential for accelerating music creation and performance by improving the tools for interacting with music software on a laptop, and thereby exceeding the capabilities of music production on dedicated music hardware \tdr{lidt søndagskringlet}.




% Electronic music
% Electronic music is a general term used for music that utilizes electronic instruments for sound generation, as opposed to electromechanichal instruments such as the electric guitar, where the purpose of the electronic circuitry is nothing but recording and playing back.\\ The class of electronic instruments is though ever expanding along with the technological development; starting with manipulating magnetic tape recordings. Later the introduction of oscillating analogue circuitry revolutionized electronic music composition, and latest digital instruments, software DSP tools and Digital Audio Workstations(DAW's) today has again had a huge impact
% \tdr{kilde}
% Alongside the rapid evolution of the general purpose laptop, this has become the instrument of choice for most musicians today, suiting both the purpose of composing and performing. 

% This project aims to ease the human-computer interaction for musical purposes, by designing a physical interface that harmonizes with music cognition, in collaboration with the company \textit{Kenton Productions}.

% \tdr{indled citat - the composers role in electronic music}

% \tdr{bro mellem state-of-the-art og vores projekt}



% OUTLINE
% introduktion
%   rhythm sequencing hvad er det
%   motivation for intuitive
%   paper: traditional rhythms
%   implementationer i max og supercollider -> dette projekt fokuserer på et intuitivt interface der gør algoritmen tilgængelig for musikere uden programmeringserfaring
%   længere, eksempler

% Analysis
%   Euclidean algorithm
%       forklaring algoritmen i detaljer
            % steps pulses osv. slicing
            % traditionelle rytmer polyrytmik
            % kompleks system, grafisk repræsentation geometri
            
%   Human computer interaction
%       cognitive dissonance theory
%       cross modal interactions
%       tradition indenfor elektroniske musik instrumenter
%           laptop limitations
%           gestural 
%       hvorfor laver vi hardware
