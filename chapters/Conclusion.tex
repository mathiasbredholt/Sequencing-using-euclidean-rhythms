%!TEX root = ../Thesis.tex
\chapter{Conclusion}

% is implemented as hardware with a physical user interface

% 2.has programmable playback only and no recording capabilities

% 3.is meant for programming rhythms (melodies are out of scope)

% 4.communicates with software audio modules only

% 5.has time resolution of 96 parts per quarter note

% 6.generates musical events using euclidean rhythms with the following additions:
%     a)possibility of rhythms with dynamic onset-density
%     b)possibility of expressive timingc)possibility of dynamic accentuation

% 7.features a graphical cyclic representation

In this project, a sequencer using euclidean rhythms has been designed as a physical interface for producing music with software audio modules. An analysis of Bjorklund's algorithm for generating euclidean rhythms was carried out in order to find the strength and weaknesses with this approach for generating rhythms. Two major limitations were found, which were not found satisfactory for a sequencer design. A solution  involving a groove system and the addition of parameters to the algorithm were found to overcome these limitations. This solution were implemented into the design. Timing requirements of the platform were specified using literature about human perception of rhythms. The design were implemented as a combination of a hardware platform and a cross-platform app communicating via WiFi. The hardware platform featured a simple user interface with rotary encoders and a keypad. The app featured a graphical representation of euclidean rhythms and an overview of the parameters. Through measurement of the sequencer output it was concluded that the timing requirements were met. The sequencer was evaluated by 26 people, who filled out a survey. The survey showed that \tdr{resultat spørgeskema}
Many additional features has been proposed that could expand the possibilities of the sequencer. 

\tdr{noget med introduktionen}